
% Default to the notebook output style

    


% Inherit from the specified cell style.




    
\documentclass[11pt]{article}

    
    
    \usepackage[T1]{fontenc}
    % Nicer default font (+ math font) than Computer Modern for most use cases
    \usepackage{mathpazo}

    % Basic figure setup, for now with no caption control since it's done
    % automatically by Pandoc (which extracts ![](path) syntax from Markdown).
    \usepackage{graphicx}
    % We will generate all images so they have a width \maxwidth. This means
    % that they will get their normal width if they fit onto the page, but
    % are scaled down if they would overflow the margins.
    \makeatletter
    \def\maxwidth{\ifdim\Gin@nat@width>\linewidth\linewidth
    \else\Gin@nat@width\fi}
    \makeatother
    \let\Oldincludegraphics\includegraphics
    % Set max figure width to be 80% of text width, for now hardcoded.
    \renewcommand{\includegraphics}[1]{\Oldincludegraphics[width=.8\maxwidth]{#1}}
    % Ensure that by default, figures have no caption (until we provide a
    % proper Figure object with a Caption API and a way to capture that
    % in the conversion process - todo).
    \usepackage{caption}
    \DeclareCaptionLabelFormat{nolabel}{}
    \captionsetup{labelformat=nolabel}

    \usepackage{adjustbox} % Used to constrain images to a maximum size 
    \usepackage{xcolor} % Allow colors to be defined
    \usepackage{enumerate} % Needed for markdown enumerations to work
    \usepackage{geometry} % Used to adjust the document margins
    \usepackage{amsmath} % Equations
    \usepackage{amssymb} % Equations
    \usepackage{textcomp} % defines textquotesingle
    % Hack from http://tex.stackexchange.com/a/47451/13684:
    \AtBeginDocument{%
        \def\PYZsq{\textquotesingle}% Upright quotes in Pygmentized code
    }
    \usepackage{upquote} % Upright quotes for verbatim code
    \usepackage{eurosym} % defines \euro
    \usepackage[mathletters]{ucs} % Extended unicode (utf-8) support
    \usepackage[utf8x]{inputenc} % Allow utf-8 characters in the tex document
    \usepackage{fancyvrb} % verbatim replacement that allows latex
    \usepackage{grffile} % extends the file name processing of package graphics 
                         % to support a larger range 
    % The hyperref package gives us a pdf with properly built
    % internal navigation ('pdf bookmarks' for the table of contents,
    % internal cross-reference links, web links for URLs, etc.)
    \usepackage{hyperref}
    \usepackage{longtable} % longtable support required by pandoc >1.10
    \usepackage{booktabs}  % table support for pandoc > 1.12.2
    \usepackage[inline]{enumitem} % IRkernel/repr support (it uses the enumerate* environment)
    \usepackage[normalem]{ulem} % ulem is needed to support strikethroughs (\sout)
                                % normalem makes italics be italics, not underlines
    

    
    
    % Colors for the hyperref package
    \definecolor{urlcolor}{rgb}{0,.145,.698}
    \definecolor{linkcolor}{rgb}{.71,0.21,0.01}
    \definecolor{citecolor}{rgb}{.12,.54,.11}

    % ANSI colors
    \definecolor{ansi-black}{HTML}{3E424D}
    \definecolor{ansi-black-intense}{HTML}{282C36}
    \definecolor{ansi-red}{HTML}{E75C58}
    \definecolor{ansi-red-intense}{HTML}{B22B31}
    \definecolor{ansi-green}{HTML}{00A250}
    \definecolor{ansi-green-intense}{HTML}{007427}
    \definecolor{ansi-yellow}{HTML}{DDB62B}
    \definecolor{ansi-yellow-intense}{HTML}{B27D12}
    \definecolor{ansi-blue}{HTML}{208FFB}
    \definecolor{ansi-blue-intense}{HTML}{0065CA}
    \definecolor{ansi-magenta}{HTML}{D160C4}
    \definecolor{ansi-magenta-intense}{HTML}{A03196}
    \definecolor{ansi-cyan}{HTML}{60C6C8}
    \definecolor{ansi-cyan-intense}{HTML}{258F8F}
    \definecolor{ansi-white}{HTML}{C5C1B4}
    \definecolor{ansi-white-intense}{HTML}{A1A6B2}

    % commands and environments needed by pandoc snippets
    % extracted from the output of `pandoc -s`
    \providecommand{\tightlist}{%
      \setlength{\itemsep}{0pt}\setlength{\parskip}{0pt}}
    \DefineVerbatimEnvironment{Highlighting}{Verbatim}{commandchars=\\\{\}}
    % Add ',fontsize=\small' for more characters per line
    \newenvironment{Shaded}{}{}
    \newcommand{\KeywordTok}[1]{\textcolor[rgb]{0.00,0.44,0.13}{\textbf{{#1}}}}
    \newcommand{\DataTypeTok}[1]{\textcolor[rgb]{0.56,0.13,0.00}{{#1}}}
    \newcommand{\DecValTok}[1]{\textcolor[rgb]{0.25,0.63,0.44}{{#1}}}
    \newcommand{\BaseNTok}[1]{\textcolor[rgb]{0.25,0.63,0.44}{{#1}}}
    \newcommand{\FloatTok}[1]{\textcolor[rgb]{0.25,0.63,0.44}{{#1}}}
    \newcommand{\CharTok}[1]{\textcolor[rgb]{0.25,0.44,0.63}{{#1}}}
    \newcommand{\StringTok}[1]{\textcolor[rgb]{0.25,0.44,0.63}{{#1}}}
    \newcommand{\CommentTok}[1]{\textcolor[rgb]{0.38,0.63,0.69}{\textit{{#1}}}}
    \newcommand{\OtherTok}[1]{\textcolor[rgb]{0.00,0.44,0.13}{{#1}}}
    \newcommand{\AlertTok}[1]{\textcolor[rgb]{1.00,0.00,0.00}{\textbf{{#1}}}}
    \newcommand{\FunctionTok}[1]{\textcolor[rgb]{0.02,0.16,0.49}{{#1}}}
    \newcommand{\RegionMarkerTok}[1]{{#1}}
    \newcommand{\ErrorTok}[1]{\textcolor[rgb]{1.00,0.00,0.00}{\textbf{{#1}}}}
    \newcommand{\NormalTok}[1]{{#1}}
    
    % Additional commands for more recent versions of Pandoc
    \newcommand{\ConstantTok}[1]{\textcolor[rgb]{0.53,0.00,0.00}{{#1}}}
    \newcommand{\SpecialCharTok}[1]{\textcolor[rgb]{0.25,0.44,0.63}{{#1}}}
    \newcommand{\VerbatimStringTok}[1]{\textcolor[rgb]{0.25,0.44,0.63}{{#1}}}
    \newcommand{\SpecialStringTok}[1]{\textcolor[rgb]{0.73,0.40,0.53}{{#1}}}
    \newcommand{\ImportTok}[1]{{#1}}
    \newcommand{\DocumentationTok}[1]{\textcolor[rgb]{0.73,0.13,0.13}{\textit{{#1}}}}
    \newcommand{\AnnotationTok}[1]{\textcolor[rgb]{0.38,0.63,0.69}{\textbf{\textit{{#1}}}}}
    \newcommand{\CommentVarTok}[1]{\textcolor[rgb]{0.38,0.63,0.69}{\textbf{\textit{{#1}}}}}
    \newcommand{\VariableTok}[1]{\textcolor[rgb]{0.10,0.09,0.49}{{#1}}}
    \newcommand{\ControlFlowTok}[1]{\textcolor[rgb]{0.00,0.44,0.13}{\textbf{{#1}}}}
    \newcommand{\OperatorTok}[1]{\textcolor[rgb]{0.40,0.40,0.40}{{#1}}}
    \newcommand{\BuiltInTok}[1]{{#1}}
    \newcommand{\ExtensionTok}[1]{{#1}}
    \newcommand{\PreprocessorTok}[1]{\textcolor[rgb]{0.74,0.48,0.00}{{#1}}}
    \newcommand{\AttributeTok}[1]{\textcolor[rgb]{0.49,0.56,0.16}{{#1}}}
    \newcommand{\InformationTok}[1]{\textcolor[rgb]{0.38,0.63,0.69}{\textbf{\textit{{#1}}}}}
    \newcommand{\WarningTok}[1]{\textcolor[rgb]{0.38,0.63,0.69}{\textbf{\textit{{#1}}}}}
    
    
    % Define a nice break command that doesn't care if a line doesn't already
    % exist.
    \def\br{\hspace*{\fill} \\* }
    % Math Jax compatability definitions
    \def\gt{>}
    \def\lt{<}
    % Document parameters
    \title{TimeSeries\_FOL\_RNN-V}
    
    
    

    % Pygments definitions
    
\makeatletter
\def\PY@reset{\let\PY@it=\relax \let\PY@bf=\relax%
    \let\PY@ul=\relax \let\PY@tc=\relax%
    \let\PY@bc=\relax \let\PY@ff=\relax}
\def\PY@tok#1{\csname PY@tok@#1\endcsname}
\def\PY@toks#1+{\ifx\relax#1\empty\else%
    \PY@tok{#1}\expandafter\PY@toks\fi}
\def\PY@do#1{\PY@bc{\PY@tc{\PY@ul{%
    \PY@it{\PY@bf{\PY@ff{#1}}}}}}}
\def\PY#1#2{\PY@reset\PY@toks#1+\relax+\PY@do{#2}}

\expandafter\def\csname PY@tok@kn\endcsname{\let\PY@bf=\textbf\def\PY@tc##1{\textcolor[rgb]{0.00,0.50,0.00}{##1}}}
\expandafter\def\csname PY@tok@ow\endcsname{\let\PY@bf=\textbf\def\PY@tc##1{\textcolor[rgb]{0.67,0.13,1.00}{##1}}}
\expandafter\def\csname PY@tok@nn\endcsname{\let\PY@bf=\textbf\def\PY@tc##1{\textcolor[rgb]{0.00,0.00,1.00}{##1}}}
\expandafter\def\csname PY@tok@na\endcsname{\def\PY@tc##1{\textcolor[rgb]{0.49,0.56,0.16}{##1}}}
\expandafter\def\csname PY@tok@sa\endcsname{\def\PY@tc##1{\textcolor[rgb]{0.73,0.13,0.13}{##1}}}
\expandafter\def\csname PY@tok@nl\endcsname{\def\PY@tc##1{\textcolor[rgb]{0.63,0.63,0.00}{##1}}}
\expandafter\def\csname PY@tok@il\endcsname{\def\PY@tc##1{\textcolor[rgb]{0.40,0.40,0.40}{##1}}}
\expandafter\def\csname PY@tok@go\endcsname{\def\PY@tc##1{\textcolor[rgb]{0.53,0.53,0.53}{##1}}}
\expandafter\def\csname PY@tok@s\endcsname{\def\PY@tc##1{\textcolor[rgb]{0.73,0.13,0.13}{##1}}}
\expandafter\def\csname PY@tok@bp\endcsname{\def\PY@tc##1{\textcolor[rgb]{0.00,0.50,0.00}{##1}}}
\expandafter\def\csname PY@tok@gt\endcsname{\def\PY@tc##1{\textcolor[rgb]{0.00,0.27,0.87}{##1}}}
\expandafter\def\csname PY@tok@c\endcsname{\let\PY@it=\textit\def\PY@tc##1{\textcolor[rgb]{0.25,0.50,0.50}{##1}}}
\expandafter\def\csname PY@tok@cs\endcsname{\let\PY@it=\textit\def\PY@tc##1{\textcolor[rgb]{0.25,0.50,0.50}{##1}}}
\expandafter\def\csname PY@tok@kr\endcsname{\let\PY@bf=\textbf\def\PY@tc##1{\textcolor[rgb]{0.00,0.50,0.00}{##1}}}
\expandafter\def\csname PY@tok@w\endcsname{\def\PY@tc##1{\textcolor[rgb]{0.73,0.73,0.73}{##1}}}
\expandafter\def\csname PY@tok@ss\endcsname{\def\PY@tc##1{\textcolor[rgb]{0.10,0.09,0.49}{##1}}}
\expandafter\def\csname PY@tok@s2\endcsname{\def\PY@tc##1{\textcolor[rgb]{0.73,0.13,0.13}{##1}}}
\expandafter\def\csname PY@tok@mb\endcsname{\def\PY@tc##1{\textcolor[rgb]{0.40,0.40,0.40}{##1}}}
\expandafter\def\csname PY@tok@ch\endcsname{\let\PY@it=\textit\def\PY@tc##1{\textcolor[rgb]{0.25,0.50,0.50}{##1}}}
\expandafter\def\csname PY@tok@gs\endcsname{\let\PY@bf=\textbf}
\expandafter\def\csname PY@tok@dl\endcsname{\def\PY@tc##1{\textcolor[rgb]{0.73,0.13,0.13}{##1}}}
\expandafter\def\csname PY@tok@nf\endcsname{\def\PY@tc##1{\textcolor[rgb]{0.00,0.00,1.00}{##1}}}
\expandafter\def\csname PY@tok@si\endcsname{\let\PY@bf=\textbf\def\PY@tc##1{\textcolor[rgb]{0.73,0.40,0.53}{##1}}}
\expandafter\def\csname PY@tok@sx\endcsname{\def\PY@tc##1{\textcolor[rgb]{0.00,0.50,0.00}{##1}}}
\expandafter\def\csname PY@tok@gd\endcsname{\def\PY@tc##1{\textcolor[rgb]{0.63,0.00,0.00}{##1}}}
\expandafter\def\csname PY@tok@nv\endcsname{\def\PY@tc##1{\textcolor[rgb]{0.10,0.09,0.49}{##1}}}
\expandafter\def\csname PY@tok@nc\endcsname{\let\PY@bf=\textbf\def\PY@tc##1{\textcolor[rgb]{0.00,0.00,1.00}{##1}}}
\expandafter\def\csname PY@tok@sc\endcsname{\def\PY@tc##1{\textcolor[rgb]{0.73,0.13,0.13}{##1}}}
\expandafter\def\csname PY@tok@mo\endcsname{\def\PY@tc##1{\textcolor[rgb]{0.40,0.40,0.40}{##1}}}
\expandafter\def\csname PY@tok@ge\endcsname{\let\PY@it=\textit}
\expandafter\def\csname PY@tok@gi\endcsname{\def\PY@tc##1{\textcolor[rgb]{0.00,0.63,0.00}{##1}}}
\expandafter\def\csname PY@tok@no\endcsname{\def\PY@tc##1{\textcolor[rgb]{0.53,0.00,0.00}{##1}}}
\expandafter\def\csname PY@tok@vc\endcsname{\def\PY@tc##1{\textcolor[rgb]{0.10,0.09,0.49}{##1}}}
\expandafter\def\csname PY@tok@kp\endcsname{\def\PY@tc##1{\textcolor[rgb]{0.00,0.50,0.00}{##1}}}
\expandafter\def\csname PY@tok@ni\endcsname{\let\PY@bf=\textbf\def\PY@tc##1{\textcolor[rgb]{0.60,0.60,0.60}{##1}}}
\expandafter\def\csname PY@tok@gp\endcsname{\let\PY@bf=\textbf\def\PY@tc##1{\textcolor[rgb]{0.00,0.00,0.50}{##1}}}
\expandafter\def\csname PY@tok@vg\endcsname{\def\PY@tc##1{\textcolor[rgb]{0.10,0.09,0.49}{##1}}}
\expandafter\def\csname PY@tok@se\endcsname{\let\PY@bf=\textbf\def\PY@tc##1{\textcolor[rgb]{0.73,0.40,0.13}{##1}}}
\expandafter\def\csname PY@tok@kc\endcsname{\let\PY@bf=\textbf\def\PY@tc##1{\textcolor[rgb]{0.00,0.50,0.00}{##1}}}
\expandafter\def\csname PY@tok@sb\endcsname{\def\PY@tc##1{\textcolor[rgb]{0.73,0.13,0.13}{##1}}}
\expandafter\def\csname PY@tok@o\endcsname{\def\PY@tc##1{\textcolor[rgb]{0.40,0.40,0.40}{##1}}}
\expandafter\def\csname PY@tok@ne\endcsname{\let\PY@bf=\textbf\def\PY@tc##1{\textcolor[rgb]{0.82,0.25,0.23}{##1}}}
\expandafter\def\csname PY@tok@vm\endcsname{\def\PY@tc##1{\textcolor[rgb]{0.10,0.09,0.49}{##1}}}
\expandafter\def\csname PY@tok@nd\endcsname{\def\PY@tc##1{\textcolor[rgb]{0.67,0.13,1.00}{##1}}}
\expandafter\def\csname PY@tok@sr\endcsname{\def\PY@tc##1{\textcolor[rgb]{0.73,0.40,0.53}{##1}}}
\expandafter\def\csname PY@tok@err\endcsname{\def\PY@bc##1{\setlength{\fboxsep}{0pt}\fcolorbox[rgb]{1.00,0.00,0.00}{1,1,1}{\strut ##1}}}
\expandafter\def\csname PY@tok@sd\endcsname{\let\PY@it=\textit\def\PY@tc##1{\textcolor[rgb]{0.73,0.13,0.13}{##1}}}
\expandafter\def\csname PY@tok@m\endcsname{\def\PY@tc##1{\textcolor[rgb]{0.40,0.40,0.40}{##1}}}
\expandafter\def\csname PY@tok@nt\endcsname{\let\PY@bf=\textbf\def\PY@tc##1{\textcolor[rgb]{0.00,0.50,0.00}{##1}}}
\expandafter\def\csname PY@tok@gh\endcsname{\let\PY@bf=\textbf\def\PY@tc##1{\textcolor[rgb]{0.00,0.00,0.50}{##1}}}
\expandafter\def\csname PY@tok@cm\endcsname{\let\PY@it=\textit\def\PY@tc##1{\textcolor[rgb]{0.25,0.50,0.50}{##1}}}
\expandafter\def\csname PY@tok@mh\endcsname{\def\PY@tc##1{\textcolor[rgb]{0.40,0.40,0.40}{##1}}}
\expandafter\def\csname PY@tok@sh\endcsname{\def\PY@tc##1{\textcolor[rgb]{0.73,0.13,0.13}{##1}}}
\expandafter\def\csname PY@tok@mi\endcsname{\def\PY@tc##1{\textcolor[rgb]{0.40,0.40,0.40}{##1}}}
\expandafter\def\csname PY@tok@c1\endcsname{\let\PY@it=\textit\def\PY@tc##1{\textcolor[rgb]{0.25,0.50,0.50}{##1}}}
\expandafter\def\csname PY@tok@kd\endcsname{\let\PY@bf=\textbf\def\PY@tc##1{\textcolor[rgb]{0.00,0.50,0.00}{##1}}}
\expandafter\def\csname PY@tok@vi\endcsname{\def\PY@tc##1{\textcolor[rgb]{0.10,0.09,0.49}{##1}}}
\expandafter\def\csname PY@tok@cpf\endcsname{\let\PY@it=\textit\def\PY@tc##1{\textcolor[rgb]{0.25,0.50,0.50}{##1}}}
\expandafter\def\csname PY@tok@mf\endcsname{\def\PY@tc##1{\textcolor[rgb]{0.40,0.40,0.40}{##1}}}
\expandafter\def\csname PY@tok@gu\endcsname{\let\PY@bf=\textbf\def\PY@tc##1{\textcolor[rgb]{0.50,0.00,0.50}{##1}}}
\expandafter\def\csname PY@tok@fm\endcsname{\def\PY@tc##1{\textcolor[rgb]{0.00,0.00,1.00}{##1}}}
\expandafter\def\csname PY@tok@gr\endcsname{\def\PY@tc##1{\textcolor[rgb]{1.00,0.00,0.00}{##1}}}
\expandafter\def\csname PY@tok@s1\endcsname{\def\PY@tc##1{\textcolor[rgb]{0.73,0.13,0.13}{##1}}}
\expandafter\def\csname PY@tok@kt\endcsname{\def\PY@tc##1{\textcolor[rgb]{0.69,0.00,0.25}{##1}}}
\expandafter\def\csname PY@tok@cp\endcsname{\def\PY@tc##1{\textcolor[rgb]{0.74,0.48,0.00}{##1}}}
\expandafter\def\csname PY@tok@nb\endcsname{\def\PY@tc##1{\textcolor[rgb]{0.00,0.50,0.00}{##1}}}
\expandafter\def\csname PY@tok@k\endcsname{\let\PY@bf=\textbf\def\PY@tc##1{\textcolor[rgb]{0.00,0.50,0.00}{##1}}}

\def\PYZbs{\char`\\}
\def\PYZus{\char`\_}
\def\PYZob{\char`\{}
\def\PYZcb{\char`\}}
\def\PYZca{\char`\^}
\def\PYZam{\char`\&}
\def\PYZlt{\char`\<}
\def\PYZgt{\char`\>}
\def\PYZsh{\char`\#}
\def\PYZpc{\char`\%}
\def\PYZdl{\char`\$}
\def\PYZhy{\char`\-}
\def\PYZsq{\char`\'}
\def\PYZdq{\char`\"}
\def\PYZti{\char`\~}
% for compatibility with earlier versions
\def\PYZat{@}
\def\PYZlb{[}
\def\PYZrb{]}
\makeatother


    % Exact colors from NB
    \definecolor{incolor}{rgb}{0.0, 0.0, 0.5}
    \definecolor{outcolor}{rgb}{0.545, 0.0, 0.0}



    
    % Prevent overflowing lines due to hard-to-break entities
    \sloppy 
    % Setup hyperref package
    \hypersetup{
      breaklinks=true,  % so long urls are correctly broken across lines
      colorlinks=true,
      urlcolor=urlcolor,
      linkcolor=linkcolor,
      citecolor=citecolor,
      }
    % Slightly bigger margins than the latex defaults
    
    \geometry{verbose,tmargin=1in,bmargin=1in,lmargin=1in,rmargin=1in}
    
    

    \begin{document}
    
    
    \maketitle
    
    

    
    \hypertarget{perform-time-series-prediction---rnn}{%
\section{Perform time series prediction -
RNN}\label{perform-time-series-prediction---rnn}}

In this project you will perform time series prediction using a
Recurrent Neural Network regressor. In particular you will re-create the
figure shown in the notes - where the stock price of Apple was
forecasted (or predicted) 7 days in advance.

The particular network architecture we will employ for our RNN is known
as \href{https://en.wikipedia.org/wiki/Long_short-term_memory}{Long Term
Short Memory (LSTM)}, which helps significantly avoid technical problems
with optimization of RNNs.

    \begin{Verbatim}[commandchars=\\\{\}]
{\color{incolor}In [{\color{incolor}17}]:} \PY{c+c1}{\PYZsh{}\PYZsh{}\PYZsh{} Load in necessary libraries for data input and normalization}
         \PY{o}{\PYZpc{}}\PY{k}{matplotlib} inline
         \PY{o}{\PYZpc{}}\PY{k}{load\PYZus{}ext} autoreload
         \PY{o}{\PYZpc{}}\PY{k}{autoreload} 2
         
         \PY{k+kn}{import} \PY{n+nn}{numpy} \PY{k}{as} \PY{n+nn}{np}
         \PY{k+kn}{import} \PY{n+nn}{matplotlib}\PY{n+nn}{.}\PY{n+nn}{pyplot} \PY{k}{as} \PY{n+nn}{plt}
         \PY{k+kn}{from} \PY{n+nn}{my\PYZus{}answers} \PY{k}{import} \PY{o}{*}
         
         \PY{c+c1}{\PYZsh{}\PYZsh{}\PYZsh{} load in and normalize the dataset}
         \PY{n}{dataset\PYZus{}Ori} \PY{o}{=} \PY{n}{np}\PY{o}{.}\PY{n}{loadtxt}\PY{p}{(}\PY{l+s+s1}{\PYZsq{}}\PY{l+s+s1}{/home/johan/repos/GitHub/Time\PYZhy{}SeriesForecasting/Data/apple\PYZus{}prices.csv}\PY{l+s+s1}{\PYZsq{}}\PY{p}{)}
         \PY{n}{dataset} \PY{o}{=} \PY{n}{np}\PY{o}{.}\PY{n}{loadtxt}\PY{p}{(}\PY{l+s+s1}{\PYZsq{}}\PY{l+s+s1}{/home/johan/repos/GitHub/Time\PYZhy{}SeriesForecasting/Data/normalized\PYZus{}apple\PYZus{}prices.csv}\PY{l+s+s1}{\PYZsq{}}\PY{p}{)}
\end{Verbatim}


    \begin{Verbatim}[commandchars=\\\{\}]
The autoreload extension is already loaded. To reload it, use:
  \%reload\_ext autoreload

    \end{Verbatim}

    \begin{Verbatim}[commandchars=\\\{\}]
{\color{incolor}In [{\color{incolor}2}]:} \PY{c+c1}{\PYZsh{} lets take a look at our time series}
        \PY{n}{plt}\PY{o}{.}\PY{n}{plot}\PY{p}{(}\PY{n}{dataset\PYZus{}Ori}\PY{p}{)}
        \PY{n}{plt}\PY{o}{.}\PY{n}{xlabel}\PY{p}{(}\PY{l+s+s1}{\PYZsq{}}\PY{l+s+s1}{time period}\PY{l+s+s1}{\PYZsq{}}\PY{p}{)}
        \PY{n}{plt}\PY{o}{.}\PY{n}{ylabel}\PY{p}{(}\PY{l+s+s1}{\PYZsq{}}\PY{l+s+s1}{normalized series value}\PY{l+s+s1}{\PYZsq{}}\PY{p}{)}
\end{Verbatim}


\begin{Verbatim}[commandchars=\\\{\}]
{\color{outcolor}Out[{\color{outcolor}2}]:} <matplotlib.text.Text at 0x7f9d70d94fd0>
\end{Verbatim}
            
    \begin{center}
    \adjustimage{max size={0.9\linewidth}{0.9\paperheight}}{output_2_1.png}
    \end{center}
    { \hspace*{\fill} \\}
    
    Lets take a quick look at the (normalized) time series we'll be
performing predictions on.

    \begin{Verbatim}[commandchars=\\\{\}]
{\color{incolor}In [{\color{incolor}3}]:} \PY{c+c1}{\PYZsh{} lets take a look at our time series}
        \PY{n}{plt}\PY{o}{.}\PY{n}{plot}\PY{p}{(}\PY{n}{dataset}\PY{p}{)}
        \PY{n}{plt}\PY{o}{.}\PY{n}{xlabel}\PY{p}{(}\PY{l+s+s1}{\PYZsq{}}\PY{l+s+s1}{time period}\PY{l+s+s1}{\PYZsq{}}\PY{p}{)}
        \PY{n}{plt}\PY{o}{.}\PY{n}{ylabel}\PY{p}{(}\PY{l+s+s1}{\PYZsq{}}\PY{l+s+s1}{normalized series value}\PY{l+s+s1}{\PYZsq{}}\PY{p}{)}
\end{Verbatim}


\begin{Verbatim}[commandchars=\\\{\}]
{\color{outcolor}Out[{\color{outcolor}3}]:} <matplotlib.text.Text at 0x7f9d6ecb5e80>
\end{Verbatim}
            
    \begin{center}
    \adjustimage{max size={0.9\linewidth}{0.9\paperheight}}{output_4_1.png}
    \end{center}
    { \hspace*{\fill} \\}
    
    \textbf{Cutting our time series into sequences}

Remember, our time series is a sequence of numbers that we can represent
in general mathematically as

\[s_{0},s_{1},s_{2},...,s_{P}\]

where \(s_{p}\) is the numerical value of the time series at time period
\(p\) and where \(P\) is the total length of the series. In order to
apply our RNN I treat the time series prediction problem as a regression
problem, and so need to use a sliding window to construct a set of
associated input/output pairs to regress on.

For example - using a window of size T = 5 (as illustrated in the gif
above) I produce a set of input/output pairs like the one shown in the
table below

\[\begin{array}{c|c}
\text{Input} & \text{Output}\\
\hline \color{CornflowerBlue} {\langle s_{1},s_{2},s_{3},s_{4},s_{5}\rangle} & \color{Goldenrod}{ s_{6}} \\
\ \color{CornflowerBlue} {\langle s_{2},s_{3},s_{4},s_{5},s_{6} \rangle } & \color{Goldenrod} {s_{7} } \\
\color{CornflowerBlue}  {\vdots} & \color{Goldenrod} {\vdots}\\
\color{CornflowerBlue} { \langle s_{P-5},s_{P-4},s_{P-3},s_{P-2},s_{P-1} \rangle } & \color{Goldenrod} {s_{P}}
\end{array}\]

Notice here that each input is a sequence (or vector) of length 5 (and
in general has length equal to the window size T) while each
corresponding output is a scalar value. Notice also how given a time
series of length P and window size T = 5 as shown above, we created P -
5 input/output pairs. More generally, for a window size T we create P -
T such pairs.

    \begin{Verbatim}[commandchars=\\\{\}]
{\color{incolor}In [{\color{incolor}20}]:} \PY{c+c1}{\PYZsh{}You can test your function on the list of odd numbers given below}
         \PY{n}{odd\PYZus{}nums} \PY{o}{=} \PY{n}{np}\PY{o}{.}\PY{n}{array}\PY{p}{(}\PY{p}{[}\PY{l+m+mi}{1}\PY{p}{,}\PY{l+m+mi}{3}\PY{p}{,}\PY{l+m+mi}{5}\PY{p}{,}\PY{l+m+mi}{7}\PY{p}{,}\PY{l+m+mi}{9}\PY{p}{,}\PY{l+m+mi}{11}\PY{p}{,}\PY{l+m+mi}{13}\PY{p}{]}\PY{p}{)}
\end{Verbatim}


    Here is a hard-coded solution for odd\_nums. You can compare its results
with what you get from your \textbf{window\_transform\_series}
implementation.

    \begin{Verbatim}[commandchars=\\\{\}]
{\color{incolor}In [{\color{incolor}5}]:} \PY{c+c1}{\PYZsh{} run a window of size 2 over the odd number sequence and display the results}
        \PY{n}{window\PYZus{}size} \PY{o}{=} \PY{l+m+mi}{2}
        
        \PY{n}{X} \PY{o}{=} \PY{p}{[}\PY{p}{]}
        \PY{n}{X}\PY{o}{.}\PY{n}{append}\PY{p}{(}\PY{n}{odd\PYZus{}nums}\PY{p}{[}\PY{l+m+mi}{0}\PY{p}{:}\PY{l+m+mi}{2}\PY{p}{]}\PY{p}{)}
        \PY{n}{X}\PY{o}{.}\PY{n}{append}\PY{p}{(}\PY{n}{odd\PYZus{}nums}\PY{p}{[}\PY{l+m+mi}{1}\PY{p}{:}\PY{l+m+mi}{3}\PY{p}{]}\PY{p}{)}
        \PY{n}{X}\PY{o}{.}\PY{n}{append}\PY{p}{(}\PY{n}{odd\PYZus{}nums}\PY{p}{[}\PY{l+m+mi}{2}\PY{p}{:}\PY{l+m+mi}{4}\PY{p}{]}\PY{p}{)}
        \PY{n}{X}\PY{o}{.}\PY{n}{append}\PY{p}{(}\PY{n}{odd\PYZus{}nums}\PY{p}{[}\PY{l+m+mi}{3}\PY{p}{:}\PY{l+m+mi}{5}\PY{p}{]}\PY{p}{)}
        \PY{n}{X}\PY{o}{.}\PY{n}{append}\PY{p}{(}\PY{n}{odd\PYZus{}nums}\PY{p}{[}\PY{l+m+mi}{4}\PY{p}{:}\PY{l+m+mi}{6}\PY{p}{]}\PY{p}{)}
        
        \PY{n}{y} \PY{o}{=} \PY{n}{odd\PYZus{}nums}\PY{p}{[}\PY{l+m+mi}{2}\PY{p}{:}\PY{p}{]}
        
        \PY{n}{X} \PY{o}{=} \PY{n}{np}\PY{o}{.}\PY{n}{asarray}\PY{p}{(}\PY{n}{X}\PY{p}{)}
        \PY{n}{y} \PY{o}{=} \PY{n}{np}\PY{o}{.}\PY{n}{asarray}\PY{p}{(}\PY{n}{y}\PY{p}{)}
        \PY{n}{y} \PY{o}{=} \PY{n}{np}\PY{o}{.}\PY{n}{reshape}\PY{p}{(}\PY{n}{y}\PY{p}{,} \PY{p}{(}\PY{n+nb}{len}\PY{p}{(}\PY{n}{y}\PY{p}{)}\PY{p}{,}\PY{l+m+mi}{1}\PY{p}{)}\PY{p}{)} \PY{c+c1}{\PYZsh{}optional}
        
        \PY{k}{assert}\PY{p}{(}\PY{n+nb}{type}\PY{p}{(}\PY{n}{X}\PY{p}{)}\PY{o}{.}\PY{n+nv+vm}{\PYZus{}\PYZus{}name\PYZus{}\PYZus{}} \PY{o}{==} \PY{l+s+s1}{\PYZsq{}}\PY{l+s+s1}{ndarray}\PY{l+s+s1}{\PYZsq{}}\PY{p}{)}
        \PY{k}{assert}\PY{p}{(}\PY{n+nb}{type}\PY{p}{(}\PY{n}{y}\PY{p}{)}\PY{o}{.}\PY{n+nv+vm}{\PYZus{}\PYZus{}name\PYZus{}\PYZus{}} \PY{o}{==} \PY{l+s+s1}{\PYZsq{}}\PY{l+s+s1}{ndarray}\PY{l+s+s1}{\PYZsq{}}\PY{p}{)}
        \PY{k}{assert}\PY{p}{(}\PY{n}{X}\PY{o}{.}\PY{n}{shape} \PY{o}{==} \PY{p}{(}\PY{l+m+mi}{5}\PY{p}{,}\PY{l+m+mi}{2}\PY{p}{)}\PY{p}{)}
        \PY{k}{assert}\PY{p}{(}\PY{n}{y}\PY{o}{.}\PY{n}{shape} \PY{o+ow}{in} \PY{p}{[}\PY{p}{(}\PY{l+m+mi}{5}\PY{p}{,}\PY{l+m+mi}{1}\PY{p}{)}\PY{p}{,} \PY{p}{(}\PY{l+m+mi}{5}\PY{p}{,}\PY{p}{)}\PY{p}{]}\PY{p}{)}
        
        \PY{c+c1}{\PYZsh{} print out input/output pairs \PYZhy{}\PYZhy{}\PYZgt{} here input = X, corresponding output = y}
        \PY{n+nb}{print} \PY{p}{(}\PY{l+s+s1}{\PYZsq{}}\PY{l+s+s1}{\PYZhy{}\PYZhy{}\PYZhy{} the input X will look like \PYZhy{}\PYZhy{}\PYZhy{}\PYZhy{}}\PY{l+s+s1}{\PYZsq{}}\PY{p}{)}
        \PY{n+nb}{print} \PY{p}{(}\PY{n}{X}\PY{p}{)}
        
        \PY{n+nb}{print} \PY{p}{(}\PY{l+s+s1}{\PYZsq{}}\PY{l+s+s1}{\PYZhy{}\PYZhy{}\PYZhy{} the associated output y will look like \PYZhy{}\PYZhy{}\PYZhy{}\PYZhy{}}\PY{l+s+s1}{\PYZsq{}}\PY{p}{)}
        \PY{n+nb}{print} \PY{p}{(}\PY{n}{y}\PY{p}{)}
\end{Verbatim}


    \begin{Verbatim}[commandchars=\\\{\}]
--- the input X will look like ----
[[ 1  3]
 [ 3  5]
 [ 5  7]
 [ 7  9]
 [ 9 11]]
--- the associated output y will look like ----
[[ 5]
 [ 7]
 [ 9]
 [11]
 [13]]

    \end{Verbatim}

    Again - you can check that your completed
\textbf{window\_transform\_series} function works correctly by trying it
on the odd\_nums sequence - you should get the above output.

    \begin{Verbatim}[commandchars=\\\{\}]
{\color{incolor}In [{\color{incolor}6}]:} \PY{c+c1}{\PYZsh{}\PYZsh{}\PYZsh{} DONE: implement the function window\PYZus{}transform\PYZus{}series in the file my\PYZus{}answers.py}
        \PY{k+kn}{from} \PY{n+nn}{my\PYZus{}answers} \PY{k}{import} \PY{n}{window\PYZus{}transform\PYZus{}series}
\end{Verbatim}


    With this function in place apply it to the series in the Python cell
below. We use a window\_size = 7 for these experiments.

    \begin{Verbatim}[commandchars=\\\{\}]
{\color{incolor}In [{\color{incolor}7}]:} \PY{c+c1}{\PYZsh{} window the data using your windowing function}
        \PY{n}{window\PYZus{}size} \PY{o}{=} \PY{l+m+mi}{7}
        \PY{n}{X}\PY{p}{,}\PY{n}{y} \PY{o}{=} \PY{n}{window\PYZus{}transform\PYZus{}series}\PY{p}{(}\PY{n}{series} \PY{o}{=} \PY{n}{dataset}\PY{p}{,}\PY{n}{window\PYZus{}size} \PY{o}{=} \PY{n}{window\PYZus{}size}\PY{p}{)}
\end{Verbatim}


    \textbf{Splitting into training and testing sets}

Note how here I am \textbf{not} splitting the dataset \emph{randomly} as
one typically would do when validating a regression model. This is
because our input/output pairs \emph{are related temporally}.

I want to train on one solid chunk of the series (in this case, the
first full 2/3 of it), and validate on a later chunk (the last 1/3) as
this simulates how I would predict \emph{future} values of a time
series.

    \begin{Verbatim}[commandchars=\\\{\}]
{\color{incolor}In [{\color{incolor}21}]:} \PY{c+c1}{\PYZsh{} split our dataset into training / testing sets}
         \PY{n}{train\PYZus{}test\PYZus{}split} \PY{o}{=} \PY{n+nb}{int}\PY{p}{(}\PY{n}{np}\PY{o}{.}\PY{n}{ceil}\PY{p}{(}\PY{l+m+mi}{2}\PY{o}{*}\PY{n+nb}{len}\PY{p}{(}\PY{n}{y}\PY{p}{)}\PY{o}{/}\PY{n+nb}{float}\PY{p}{(}\PY{l+m+mi}{3}\PY{p}{)}\PY{p}{)}\PY{p}{)}   \PY{c+c1}{\PYZsh{} set the split point}
         
         \PY{c+c1}{\PYZsh{} partition the training set}
         \PY{n}{X\PYZus{}train} \PY{o}{=} \PY{n}{X}\PY{p}{[}\PY{p}{:}\PY{n}{train\PYZus{}test\PYZus{}split}\PY{p}{,}\PY{p}{:}\PY{p}{]}
         \PY{n}{y\PYZus{}train} \PY{o}{=} \PY{n}{y}\PY{p}{[}\PY{p}{:}\PY{n}{train\PYZus{}test\PYZus{}split}\PY{p}{]}
         
         \PY{c+c1}{\PYZsh{} keep the last chunk for testing}
         \PY{n}{X\PYZus{}test} \PY{o}{=} \PY{n}{X}\PY{p}{[}\PY{n}{train\PYZus{}test\PYZus{}split}\PY{p}{:}\PY{p}{,}\PY{p}{:}\PY{p}{]}
         \PY{n}{y\PYZus{}test} \PY{o}{=} \PY{n}{y}\PY{p}{[}\PY{n}{train\PYZus{}test\PYZus{}split}\PY{p}{:}\PY{p}{]}
         
         \PY{c+c1}{\PYZsh{} NOTE: to use keras\PYZsq{}s RNN LSTM module our input must be reshaped to [samples, window size, stepsize] }
         \PY{n}{X\PYZus{}train} \PY{o}{=} \PY{n}{np}\PY{o}{.}\PY{n}{asarray}\PY{p}{(}\PY{n}{np}\PY{o}{.}\PY{n}{reshape}\PY{p}{(}\PY{n}{X\PYZus{}train}\PY{p}{,} \PY{p}{(}\PY{n}{X\PYZus{}train}\PY{o}{.}\PY{n}{shape}\PY{p}{[}\PY{l+m+mi}{0}\PY{p}{]}\PY{p}{,} \PY{n}{window\PYZus{}size}\PY{p}{,} \PY{l+m+mi}{1}\PY{p}{)}\PY{p}{)}\PY{p}{)}
         \PY{n}{X\PYZus{}test} \PY{o}{=} \PY{n}{np}\PY{o}{.}\PY{n}{asarray}\PY{p}{(}\PY{n}{np}\PY{o}{.}\PY{n}{reshape}\PY{p}{(}\PY{n}{X\PYZus{}test}\PY{p}{,} \PY{p}{(}\PY{n}{X\PYZus{}test}\PY{o}{.}\PY{n}{shape}\PY{p}{[}\PY{l+m+mi}{0}\PY{p}{]}\PY{p}{,} \PY{n}{window\PYZus{}size}\PY{p}{,} \PY{l+m+mi}{1}\PY{p}{)}\PY{p}{)}\PY{p}{)}
\end{Verbatim}


    \textbf{RNN regression model}

\begin{itemize}
\tightlist
\item
  layer 1 uses an LSTM module with 5 hidden units (note here the
  input\_shape = (window\_size,1))
\item
  layer 2 uses a fully connected module with one unit
\item
  the `mean\_squared\_error' loss should be used (remember: we are
  performing regression here)
\end{itemize}

This can be constructed using just a few lines - see e.g., the
\href{https://keras.io/getting-started/sequential-model-guide/}{general
Keras documentation} and the
\href{https://keras.io/layers/recurrent/}{LSTM documentation in
particular} for examples of how to quickly use Keras to build neural
network models. Make sure you are initializing your optimizer given the
\href{https://keras.io/optimizers/}{keras-recommended approach for RNNs}

    \begin{Verbatim}[commandchars=\\\{\}]
{\color{incolor}In [{\color{incolor}22}]:} \PY{c+c1}{\PYZsh{} import keras network libraries}
         \PY{k+kn}{from} \PY{n+nn}{keras}\PY{n+nn}{.}\PY{n+nn}{models} \PY{k}{import} \PY{n}{Sequential}
         \PY{k+kn}{from} \PY{n+nn}{keras}\PY{n+nn}{.}\PY{n+nn}{layers} \PY{k}{import} \PY{n}{Dense}
         \PY{k+kn}{from} \PY{n+nn}{keras}\PY{n+nn}{.}\PY{n+nn}{layers} \PY{k}{import} \PY{n}{LSTM}
         \PY{k+kn}{import} \PY{n+nn}{keras}
         \PY{k+kn}{from} \PY{n+nn}{my\PYZus{}answers} \PY{k}{import} \PY{n}{build\PYZus{}part1\PYZus{}RNN}
         
         
         \PY{c+c1}{\PYZsh{} given \PYZhy{} fix random seed \PYZhy{} so we can all reproduce the same results on our default time series}
         \PY{n}{np}\PY{o}{.}\PY{n}{random}\PY{o}{.}\PY{n}{seed}\PY{p}{(}\PY{l+m+mi}{0}\PY{p}{)}
         \PY{n}{model} \PY{o}{=} \PY{n}{build\PYZus{}part1\PYZus{}RNN}\PY{p}{(}\PY{n}{window\PYZus{}size}\PY{p}{)}
         \PY{c+c1}{\PYZsh{} build model using keras documentation recommended optimizer initialization}
         \PY{n}{optimizer} \PY{o}{=} \PY{n}{keras}\PY{o}{.}\PY{n}{optimizers}\PY{o}{.}\PY{n}{RMSprop}\PY{p}{(}\PY{n}{lr}\PY{o}{=}\PY{l+m+mf}{0.001}\PY{p}{,} \PY{n}{rho}\PY{o}{=}\PY{l+m+mf}{0.9}\PY{p}{,} \PY{n}{epsilon}\PY{o}{=}\PY{l+m+mf}{1e\PYZhy{}08}\PY{p}{,} \PY{n}{decay}\PY{o}{=}\PY{l+m+mf}{0.0}\PY{p}{)}
         \PY{c+c1}{\PYZsh{} compile the model}
         \PY{n}{model}\PY{o}{.}\PY{n}{compile}\PY{p}{(}\PY{n}{loss}\PY{o}{=}\PY{l+s+s1}{\PYZsq{}}\PY{l+s+s1}{mean\PYZus{}squared\PYZus{}error}\PY{l+s+s1}{\PYZsq{}}\PY{p}{,} \PY{n}{optimizer}\PY{o}{=}\PY{n}{optimizer}\PY{p}{)}
\end{Verbatim}


    \begin{Verbatim}[commandchars=\\\{\}]
{\color{incolor}In [{\color{incolor}10}]:} \PY{c+c1}{\PYZsh{} run\PYZhy{}model}
         \PY{n}{model}\PY{o}{.}\PY{n}{fit}\PY{p}{(}\PY{n}{X\PYZus{}train}\PY{p}{,} \PY{n}{y\PYZus{}train}\PY{p}{,} \PY{n}{epochs}\PY{o}{=}\PY{l+m+mi}{1000}\PY{p}{,} \PY{n}{batch\PYZus{}size}\PY{o}{=}\PY{l+m+mi}{50}\PY{p}{,} \PY{n}{verbose}\PY{o}{=}\PY{l+m+mi}{0}\PY{p}{)}
\end{Verbatim}


\begin{Verbatim}[commandchars=\\\{\}]
{\color{outcolor}Out[{\color{outcolor}10}]:} <keras.callbacks.History at 0x7f9d70f11d30>
\end{Verbatim}
            
    Model performance - check\\
Now I can make predictions on training and testing sets.

    \begin{Verbatim}[commandchars=\\\{\}]
{\color{incolor}In [{\color{incolor}11}]:} \PY{c+c1}{\PYZsh{} generate predictions for training}
         \PY{n}{train\PYZus{}predict} \PY{o}{=} \PY{n}{model}\PY{o}{.}\PY{n}{predict}\PY{p}{(}\PY{n}{X\PYZus{}train}\PY{p}{)}
         \PY{n}{test\PYZus{}predict} \PY{o}{=} \PY{n}{model}\PY{o}{.}\PY{n}{predict}\PY{p}{(}\PY{n}{X\PYZus{}test}\PY{p}{)}
\end{Verbatim}


    If either or both of your accuracies are larger than 0.02 re-train your
model (increasing the number of epochs you take and/or adjusting your
batch\_size).

    \begin{Verbatim}[commandchars=\\\{\}]
{\color{incolor}In [{\color{incolor}12}]:} \PY{c+c1}{\PYZsh{} print out training and testing errors}
         \PY{n}{training\PYZus{}error} \PY{o}{=} \PY{n}{model}\PY{o}{.}\PY{n}{evaluate}\PY{p}{(}\PY{n}{X\PYZus{}train}\PY{p}{,} \PY{n}{y\PYZus{}train}\PY{p}{,} \PY{n}{verbose}\PY{o}{=}\PY{l+m+mi}{0}\PY{p}{)}
         \PY{n+nb}{print}\PY{p}{(}\PY{l+s+s1}{\PYZsq{}}\PY{l+s+s1}{training error = }\PY{l+s+s1}{\PYZsq{}} \PY{o}{+} \PY{n+nb}{str}\PY{p}{(}\PY{n}{training\PYZus{}error}\PY{p}{)}\PY{p}{)}
         
         \PY{n}{testing\PYZus{}error} \PY{o}{=} \PY{n}{model}\PY{o}{.}\PY{n}{evaluate}\PY{p}{(}\PY{n}{X\PYZus{}test}\PY{p}{,} \PY{n}{y\PYZus{}test}\PY{p}{,} \PY{n}{verbose}\PY{o}{=}\PY{l+m+mi}{0}\PY{p}{)}
         \PY{n+nb}{print}\PY{p}{(}\PY{l+s+s1}{\PYZsq{}}\PY{l+s+s1}{testing error = }\PY{l+s+s1}{\PYZsq{}} \PY{o}{+} \PY{n+nb}{str}\PY{p}{(}\PY{n}{testing\PYZus{}error}\PY{p}{)}\PY{p}{)}
\end{Verbatim}


    \begin{Verbatim}[commandchars=\\\{\}]
training error = 0.016003616492856632
testing error = 0.0139842628349745

    \end{Verbatim}

    \begin{Verbatim}[commandchars=\\\{\}]
{\color{incolor}In [{\color{incolor}16}]:} \PY{c+c1}{\PYZsh{}\PYZsh{}\PYZsh{} Plot everything \PYZhy{} the original series as well as predictions on training and testing sets}
         \PY{k+kn}{import} \PY{n+nn}{matplotlib}\PY{n+nn}{.}\PY{n+nn}{pyplot} \PY{k}{as} \PY{n+nn}{plt}
         \PY{o}{\PYZpc{}}\PY{k}{matplotlib} inline
         
         \PY{c+c1}{\PYZsh{} plot original series}
         \PY{n}{plt}\PY{o}{.}\PY{n}{plot}\PY{p}{(}\PY{n}{dataset}\PY{p}{,}\PY{n}{color} \PY{o}{=} \PY{l+s+s1}{\PYZsq{}}\PY{l+s+s1}{r}\PY{l+s+s1}{\PYZsq{}}\PY{p}{)}
         
         \PY{c+c1}{\PYZsh{} plot training set prediction}
         \PY{n}{split\PYZus{}pt} \PY{o}{=} \PY{n}{train\PYZus{}test\PYZus{}split} \PY{o}{+} \PY{n}{window\PYZus{}size} 
         \PY{n}{plt}\PY{o}{.}\PY{n}{plot}\PY{p}{(}\PY{n}{np}\PY{o}{.}\PY{n}{arange}\PY{p}{(}\PY{n}{window\PYZus{}size}\PY{p}{,}\PY{n}{split\PYZus{}pt}\PY{p}{,}\PY{l+m+mi}{1}\PY{p}{)}\PY{p}{,}\PY{n}{train\PYZus{}predict}\PY{p}{,}\PY{n}{color} \PY{o}{=} \PY{l+s+s1}{\PYZsq{}}\PY{l+s+s1}{g}\PY{l+s+s1}{\PYZsq{}}\PY{p}{)}
         
         \PY{c+c1}{\PYZsh{} plot testing set prediction}
         \PY{n}{plt}\PY{o}{.}\PY{n}{plot}\PY{p}{(}\PY{n}{np}\PY{o}{.}\PY{n}{arange}\PY{p}{(}\PY{n}{split\PYZus{}pt}\PY{p}{,}\PY{n}{split\PYZus{}pt} \PY{o}{+} \PY{n+nb}{len}\PY{p}{(}\PY{n}{test\PYZus{}predict}\PY{p}{)}\PY{p}{,}\PY{l+m+mi}{1}\PY{p}{)}\PY{p}{,}\PY{n}{test\PYZus{}predict}\PY{p}{,}\PY{n}{color} \PY{o}{=} \PY{l+s+s1}{\PYZsq{}}\PY{l+s+s1}{b}\PY{l+s+s1}{\PYZsq{}}\PY{p}{)}
         
         \PY{c+c1}{\PYZsh{} pretty up graph}
         \PY{n}{plt}\PY{o}{.}\PY{n}{xlabel}\PY{p}{(}\PY{l+s+s1}{\PYZsq{}}\PY{l+s+s1}{day}\PY{l+s+s1}{\PYZsq{}}\PY{p}{)}
         \PY{n}{plt}\PY{o}{.}\PY{n}{ylabel}\PY{p}{(}\PY{l+s+s1}{\PYZsq{}}\PY{l+s+s1}{(normalized) price of Apple stock}\PY{l+s+s1}{\PYZsq{}}\PY{p}{)}
         \PY{n}{plt}\PY{o}{.}\PY{n}{legend}\PY{p}{(}\PY{p}{[}\PY{l+s+s1}{\PYZsq{}}\PY{l+s+s1}{original series}\PY{l+s+s1}{\PYZsq{}}\PY{p}{,}\PY{l+s+s1}{\PYZsq{}}\PY{l+s+s1}{training fit}\PY{l+s+s1}{\PYZsq{}}\PY{p}{,}\PY{l+s+s1}{\PYZsq{}}\PY{l+s+s1}{testing fit}\PY{l+s+s1}{\PYZsq{}}\PY{p}{]}\PY{p}{,}\PY{n}{loc}\PY{o}{=}\PY{l+s+s1}{\PYZsq{}}\PY{l+s+s1}{center left}\PY{l+s+s1}{\PYZsq{}}\PY{p}{,} \PY{n}{bbox\PYZus{}to\PYZus{}anchor}\PY{o}{=}\PY{p}{(}\PY{l+m+mi}{1}\PY{p}{,} \PY{l+m+mf}{0.5}\PY{p}{)}\PY{p}{)}
         \PY{n}{plt}\PY{o}{.}\PY{n}{show}\PY{p}{(}\PY{p}{)}
\end{Verbatim}


    \begin{center}
    \adjustimage{max size={0.9\linewidth}{0.9\paperheight}}{output_22_0.png}
    \end{center}
    { \hspace*{\fill} \\}
    

    % Add a bibliography block to the postdoc
    
    
    
    \end{document}
